\documentclass[a4paper,10pt,onecolumn,oneside,openany]{jsbook}
% 参考URL http://d.hatena.ne.jp/yosshi71jp/20101210/1292005429 感謝です
% パッケージの設定,これは不要なものもあるかもしれない
\usepackage{amsmath,amssymb}
\usepackage{bm}
\usepackage{epsbox}
\usepackage[dvipdfmx]{graphicx}
\usepackage{verbatim}
\usepackage{wrapfig}
\usepackage{ascmac}
\usepackage{makeidx}

\makeindex
%%% 余白・文字数調整(左37mm, 右18mm, 上下共30mm, 文字数約40字/行, 行数約32行)
% 実際の寸法は学科の規定に従ってくださいね
\setlength{\textwidth}{155truemm}      % テキスト幅: 210-(37+18)=155mm
\setlength{\fullwidth}{\textwidth}     % ページ全体の幅
\setlength{\oddsidemargin}{37truemm}   % 左余白
\addtolength{\oddsidemargin}{-1truein} % 左位置デフォルトから-1inch
\setlength{\topmargin}{30truemm}       % 上余白
\setlength{\textheight}{237truemm}     % テキスト高さ: 297-(30+30)=237mm
\addtolength{\topmargin}{-1truein}     % 上位置デフォルトから-1inch
% 本文の行数と桁数を指定出来るように
\def\linesparpage#1{\baselineskip=\textheight
   \divide\baselineskip by #1}
\def\kcharparline#1{%
   \ifx\xkanjiskip\undefined%
   % NTT jTeX用
   \jintercharskip 0mm plus 0.2mm minus 0.2mm
   \else
   % ASCII pTex用
   \xkanjiskip 0mm plus 0.2mm minus 0.2mm
   \fi
   \settowidth{\textwidth}{あ}%
   \multiply\textwidth by #1}
%
%
\begin{document}
\linesparpage{32} % 一ページを32行に(効果ない・・・)
\kcharparline{40} % 一行を40字に
%%% タイトル設定
\begin{titlepage}
\begin{flushright}
{\large
指導教員:来住伸子 教授 \\ % 主査
%副査:□□□□ 教授              % 副査
}
\end{flushright}
\begin{center}
\vspace{150truept}
{\huge title}\\ % タイトル
\vspace{10truept}
{\Large subtitle}\\ % サブタイトル(なければコメントアウト)
\vspace{50truept}
{\huge G13908 岩科智彩}\\ % 学籍番号 著者
\vspace{10truept}
{\huge G13924 森下汐美}\\ % 学籍番号
\vspace{50truept}
{\huge 平成29年1月13日}\\ % 提出日
\end{center}
\end{titlepage}

\frontmatter
\begin{abstract} %論文要旨
近年プログラミング教育の推進に伴い、義務教育化が進んでいる。その中で米国マサチューセッツ工科大学のメディアラボが開発したScratchは無償で提供されているグラフィックプログラミング環境である。プログラミングを行う際の命令を本ツールではブロックを組み合わせて作り上げる。初心者にとっては使いやすい構造となっているため米国では利用が増えているものの、日本のユーザーは全体の1\%にも満たない。そこで実際に本ツールで公表をされているデータを利用してより教育に用いられるツールの解析を目指す。
本ツールですでに
\begin{enumerate}
  \item 各ブロックの種類が使われている全体での割合
  \item あるプロジェクトが他ユーザーの引用関係を示したツリー構造
\end{enumerate}
が公表されている。しかしこのデータでは全体図の把握が可能であるが1つのプログラムでブロックがどのように使われているか、引用していた場合引用元からどの程度変更させたかは不明である。
従って本研究では1つのプログラムで使用されているブロックを解析し結果を出すと同時に引用元との比較を行い、関係性を導き出す。
\end{abstract}

%%% 目次
\tableofcontents
%
%
\mainmatter
%
%%% 本文ここから

\part{序論} %3部構成を取る必要がない場合もあります
\chapter{はじめに}
\section{本研究の背景}

%文章


\section{本研究の目的}

%文章

\section{本論文の構成}

%文章


%番号付きリスト
%\begin{enumerate}
% \item 『どんな課題を解決しようとしているのか?』%ここに項目を書く
%  \item 『その課題解決はうちの業務に役に立ちそうか?』
%  \item 仮に役に立たなくても『そのガンバリはうちの業務に役に立ちそうか?』
%\end{enumerate}


%箇条書きコード
%\begin{itemize} %ただの箇条書き,これも便利なので覚えるべし.
%  \item c++とC\#ではC\#のほうが得意です
%  \item Processingしか書けません
%  \item KinectはOpenNIもMicrosoft KinectSDKでもどっちでもいけます
%  \item 処理はGPU上で動いてます
%\end{itemize}

%クォーテーションはこう書く↓
%『なんだよさっき,``一番最初に''って言ったじゃないか!』とお怒りを受けそうだが,研究(\textit{re-search})を進める上で調査(\textit{search})をせよ,つまり先人の研究や論文検索をせよ,という前期のゼミで先生方が課しているようなアタリマエのことを述べただけであり,最初に書くのはせいぜいメモ程度,論文要旨程度の分量で構わない.

\part{本論}
\chapter{Scratchについて}
\section{Scratchとは}
\section{Scratchの使い方}
\section{Scratchの利点、欠点}
\section{研究におけるScratchの利用意義}
\chapter{Scratch公式可視化データの現状}
\section{ブロック}
\section{Remixツリー}
\chapter{制作内容}
\section{分析内容}
\subsection{ブロックの種類の集計}
\subsection{cos類似度の計算}
\subsection{ブロックとスプライトの合計数}
\section{グラフ化}
\section{cos類似度とブロック数のグラフ}
\section{cos類似度とスプライト数のグラフ}




%\fbox{読み手が正しく理解できるか},という品質の問題と,\fbox{読み手と同じ順序で書く}という方法は一致している必要がない.


% 表の挿入
%\begin{table}[h]
% \caption{おすすめの執筆順序}% {}内に表題を書く
% \begin{center}
%  \begin{tabular}{|c|c|l|} %セル内の位置{c:センタリング,l:左寄せ},パイプ「|」縦罫線
%    \hline
%     1  &  4章  & 開発の中身,実験方法となんとなく見えている(or 予想される)結果   \\
%    \hline
%     2  &  3章  & 理論と仮説,この段階ではだいたいでいい,上記実験のベースになっている素案ぐらいでも.   \\
%    \hline
%     3  &  5章  & 結果のグラフ,理解できるグラフを描きなおすために実験をやり直してもいい.チート厳禁.   \\
%    \hline
%     4  &  2章+3章  & 関連研究,課題設定→理論までの展開を整理しながら   \\
%    \hline
%     5  &  1章+6章  & できた結果について素直に受け止められるよう,風呂敷を広げすぎずに.   \\
%    \hline
%     6  &  論文概要  &  章構成を再度見直し,推敲時にブレないように,ここで固める.  \\
%    \hline
%     7  &  全章推敲  &  このあたりでやっと先輩や先生に見せられるレベル,ただし卒業は見通しが出る.  \\
%    \hline
%  \end{tabular}
% \end{center}
%\end{table}



% 図の挿入
%\begin{figure}[htbp]
%  \begin{center}
%    \includegraphics[bb=0 0 432 576,width=5cm]{figall/phd101212s.png}
%  \end{center}
%  \caption{Piled Higher and Deeper:「``FINAL''.doc」(10/12/2012)より http://www.phdcomics.com/ 参照}
%  \label{http://www.phdcomics.com/comics/archive.php?comicid=1531}
%\end{figure}

%\input{dq.txt} %新しいページから始める場合は\include, \inputの場合は改頁しない

\part{結論}
\chapter{評価}
\section{類似研究者}
\subsection{吉田葵先生}
\subsection{来住伸子先生}

\chapter{結論}


%%% 付録
\appendix
付録,ソースコードなど.
ソースコードを直接取り込むパッケージもある.
http://blog.santalinux.net/?p=135

%
%%% 謝辞
\chapter{謝辞}
\addcontentsline{toc}{chapter}{謝辞}
%
%
%
\chapter{参考文献}
%\bibliographystyle{sieicej} % 電子情報通信学会の論文誌スタイルになってる
%\bibliography{myrefs}

\begin{thebibliography}{99}
%\bibitem{ohno}
%大野義夫編,\TeX\ 入門,
%共立出版,東京,1989. 

%\bibitem{Seroul}
%R. Seroul and S. Levy, A Beginner's Book of \TeX, 
%Springer-Verlag, New York, 1989. 

%\bibitem{nodera1}
%野寺隆志,楽々\LaTeX{},
%共立出版,東京,1990. 

%\bibitem{itou}
%伊藤和人,\LaTeX\ トータルガイド,
%秀和システムトレーディング,1991. 

%\bibitem{nodera2}
%野寺隆志,今度こそ\AmSLaTeX{},
%共立出版,東京,1991. 

\bibitem{tex}
D.E. クヌース,改訂新版 \TeX\ ブック,
アスキー出版局,東京,1992. 

\bibitem{jiyuu}
磯崎秀樹,\LaTeX\ 自由自在,
サイエンス社,東京,1992. 

%\bibitem{impress}
%鷺谷好輝,日本語 \LaTeX\ 定番スタイル集,
%インプレス,東京,1992--1994. 

\bibitem{Bech}
S. von Bechtolsheim, \TeX\ in Practice, 
Springer-Verlag, New York, 1993. 

%\bibitem{Gr}
%G. Gr\"{a}tzer, 
%Math into \TeX\,--\,A Simple Introduction to \AmSLaTeX, 
%Birkh\"{a}user, 1993.

\bibitem{hujita}
藤田眞作,
化学者・生化学者のための\LaTeX---パソコンによる論文作成の手引,
東京化学同人,東京,1993. 

%\bibitem{styleuse}
%古川徹生,岩熊哲夫,
%\LaTeX\ のマクロやスタイルファイルの利用(styleuse.tex),1994. 

\bibitem{Ase}
阿瀬はる美,てくてく\TeX{},
アスキー出版局,東京,1994. 

\bibitem{Walsh}
N. Walsh, Making \TeX\ Work, 
O'Reilly \& Associates, Sebastopol, 1994. 

\bibitem{Salomon}
D. Salomon, The Advanced \TeX\ book, 
Springer-Verlag, New York, 1995.

\bibitem{hujita2}
藤田眞作,\LaTeX\ マクロの八衢,
アジソン・ウェスレイ・パブリッシャーズ・ジャパン,東京,1995. 

\bibitem{Nakano}
中野賢,日本語 \LaTeXe\ ブック,
アスキー出版局,東京,1996. 

\bibitem{Fujita4}
藤田眞作,\LaTeXe\ 階梯,
アジソン・ウェスレイ・パブリッシャーズ・ジャパン,東京,1996. 

\bibitem{otobe}
乙部巌己,江口庄英,
p\LaTeXe\ for Windows\ Another Manual,
ソフトバンク パブリッシング,東京,1996--1997. 

\bibitem{Abrahams}
% P.W. Abrahams, \TeX\ for the Impatient,
% (Addison-Wesley, 1992). 
ポール W. エイブラハム,明快 \TeX{},
アジソン・ウェスレイ・パブリッシャーズ・ジャパン,東京,1997. 

\bibitem{Eguchi}
江口庄英,Ghostscript Another Manual,
ソフトバンク パブリッシング,東京,1997. 

\bibitem{FMi1}
% M. Goossens, F. Mittelbach, and A. Samarin, The \LaTeX\ Companion, 
% Addison-Wesley, Reading, 1994. 
マイケル・グーセンス,フランク・ミッテルバッハ,アレキサンダー・サマリン,
\LaTeX\ コンパニオン,アスキー出版局,東京,1998. 

\bibitem{Eijkhout}
% V. Eijkhout, \TeX\ by Topic, Addison-Wesley, Wokingham, 1991. 
ビクター・エイコー,\TeX\ by Topic---\TeX\ をよく深く知るための39章,
アスキー出版局,東京,1999. 

\bibitem{latex}
%レスリー ランポート,文書処理システム\LaTeX{},
%アスキー出版局,東京,1990. 
レスリー・ランポート,文書処理システム \LaTeXe{},
ピアソンエデュケーション,東京,1999. 

\bibitem{Okumura3}
奥村晴彦,[改訂版]\LaTeXe\ 美文書作成入門,
技術評論社,東京,2000. 

\bibitem{FMi2}
% M. Goossens, S. Rahts, and  F. Mittelbach,  
% The \LaTeX\ Graphics Companion (Addison-Wesley, 1997).
マイケル・グーセンス,セバスチャン・ラッツ,フランク・ミッテルバッハ,
\LaTeX\ グラフィックスコンパニオン,アスキー出版局,東京,2000. 

\bibitem{FGo1}
% M. Goossens, and S. Rahts, 
% The \LaTeX\ Web Companion, Addison-Wesley,  1999.
マイケル・グーセンス,セバスチャン・ラッツ,
\LaTeX\ Web コンパニオン---\TeX\ とHTML/XML の統合,
アスキー出版局,東京,2001. 

\bibitem{PEn}
ページ・エンタープライゼス\<(株)\<,
\LaTeXe\ マクロ \& クラスプログラミング基礎解説,
技術評論社,東京,2002. 

\bibitem{Fujita5}
藤田眞作,\LaTeXe\ コマンドブック,
ソフトバンク パブリッシング,東京,2003. 

\bibitem{Yoshinaga}
吉永徹美,
\LaTeXe\ マクロ \& クラスプログラミング実践解説,
技術評論社,東京,2003. 
\end{thebibliography}


\newpage
\printindex
%
%
\end{document}
